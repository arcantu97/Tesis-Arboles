
\documentclass[a0,portrait]{a0poster}

\usepackage{multicol} % This is so we can have multiple columns of text side-by-side
\columnsep=100pt % This is the amount of white space between the columns in the poster
\columnseprule=3pt % This is the thickness of the black line between the columns in the poster

\usepackage[svgnames]{xcolor} % Specify colors by their 'svgnames', for a full list of all colors available see here: http://www.latextemplates.com/svgnames-colors

\usepackage{times} % Use the times font
%\usepackage{palatino} % Uncomment to use the Palatino font

\usepackage{graphicx} % Required for including images
\graphicspath{{figures/}} % Location of the graphics files
\usepackage{booktabs} % Top and bottom rules for table
\usepackage[font=small,labelfont=bf]{caption} % Required for specifying captions to tables and figures
\usepackage{amsfonts, amsmath, amsthm, amssymb} % For math fonts, symbols and environments
\usepackage{wrapfig} % Allows wrapping text around tables and figures

\begin{document}

%----------------------------------------------------------------------------------------
%	POSTER HEADER 
%----------------------------------------------------------------------------------------

% The header is divided into two boxes:
% The first is 75% wide and houses the title, subtitle, names, university/organization and contact information
% The second is 25% wide and houses a logo for your university/organization or a photo of you
% The widths of these boxes can be easily edited to accommodate your content as you see fit

\begin{minipage}[b]{1\linewidth}
\centering \veryHuge \color{Black} \textbf{Inventarios forestales a través del procesamiento de imágenes}\\ % Title
\huge \textbf{José Angel Ramírez Cantú \& Satu Elisa Schaeffer}\\[0.5cm] % Author(s)
\huge Universidad Autónoma de Nuevo León\\[0.4cm] % University/organization
\huge Facultad de Ingeniería Mecánica y Eléctrica\\[0.4cm] % University/organization
\Large \texttt{jose\_ramirezcantu@outlook.com}\\
\end{minipage}
%
\begin{minipage}[b]{0.25\linewidth}
%%\includegraphics[width=20cm]{logo.png}\\
\end{minipage}

\vspace{1cm} % A bit of extra whitespace between the header and poster content

%----------------------------------------------------------------------------------------

\begin{multicols}{2} % This is how many columns your poster will be broken into, a portrait poster is generally split into 2 columns

%----------------------------------------------------------------------------------------
%	INTRODUCTION
%----------------------------------------------------------------------------------------

\color{Black} % SaddleBrown color for the introduction

\section*{Introducción}
El \emph{aprendizaje máquina}\footnote{Campo de la inteligencia artificial que desarrolla algoritmos capaces de aprender por medio de información.} es precisamente uno de los campos de la \emph{inteligencia artificial}\footnote{Ciencia encargada de desarrollar algoritmos capaces de imitar capacidades humanas.} que permite resolver esta clase de problemas, ya que gracias al aprendizaje supervisado se pueden usar técnicas de agrupamiento para clasificar distintas especies de árboles por medio de muestras recolectadas. En análisis de recorridos por drones, el enfoque está dirigido al área forestal dado que se puede utilizar el aprendizaje máquina para la gestión del inventario forestal, ayudando a reducir el fallo humano y optimizando las tareas de clasificación de especies arbóreas.

%----------------------------------------------------------------------------------------
%	OBJECTIVES
%----------------------------------------------------------------------------------------

\section*{Motivación}
Pese a que ya existen mecanismos de detección de objetos, muchos de ellos no funcionan con la precisión o la meta que deseamos, puesto que no se enfocan en un objetivo en particular, más sin embargo, la investigación se enfoca puramente en la detección de especies arbóreas (\emph{Abies, Encino y Pino}) utilizando muestras recolectadas en las zonas del Cilantrillo y Trinidad.
%----------------------------------------------------------------------------------------
%	MATERIALS AND METHODS
%----------------------------------------------------------------------------------------

\section*{Hipótesis}
Se sabe que el procesamiento de imágenes tiene como finalidad enfocarse en la búsqueda de un elemento en particular, las especies arbóreas (presente trabajo). Se plantea demostrar que el procesamiento de imágenes permitiría reducir tiempos de recorrido a pie y optimizar costos en cuanto a la realización de inventarios forestales por medio de técnicas tradicionales.


\section*{Objetivos}
\subsection*{Objetivo general}
El objetivo de realizar el inventario forestal por medio del procesamiento de
imágenes tiene un propósito más práctico que técnico. El algoritmo permitiría a quienes se encarguen de analizar las zonas forestales, reducir el tiempo invertido en aplicar técnicas tradicionales por técnicas de procesamiento de imágenes.\\

Estas técnicas  basadas en el aprendizaje máquina, las cuales van permitir
generar un inventario forestal mediante el recorrido de un dron y a su vez, analizarlo por medio de la inteligencia artificial con la finalidad de indicar las cantidad de especies reconocidas sobre una zona.
\subsection*{Objetivos específicos}
\begin{itemize}
\item Realizar un algoritmo capaz de detectar específicamente los árboles y su especie arbórea.
\end{itemize}

\begin{itemize}
\item El algoritmo debe extraer la información de un conjunto de especies arbóreas, mismas que servirían como modelo para una fase posterior de detección de especies arbóreas.
\end{itemize}

\begin{itemize}
\item El algoritmo debe ser capaz de detectar por si mismo las especies encontradas en cada una de las muestras recolectadas.
\end{itemize}
\bigskip

\section*{Descriptores de características globales}
La idea de que el color presente en las muestras recolectadas sea el único diferenciador de una especie respecto a otra es un pensamiento incorrecto debido a que además de este criterio, existen otros criterios que permiten apreciar e identificar las características de una especie arbórea, no obstante, estas características pueden ser útiles en otras fases de la investigación. 

\subsection*{Color}
La característica de clasificación de color hace uso del \emph{histograma de color}\footnote{Cantidad de pixeles en listas de rangos de colores presentes en una imagen.}, el cual es utilizado para determinar la intensidad de un color en un \emph{pixel}\footnote{Es la unidad básica más pequeña de las imágenes.}.

\subsection*{Forma}
La característica de forma cuenta también con varias métricas, se hace énfasis en los \emph{momentos de una imagen}. 
Los momentos de una imagen son los pesos promedio de la intensidad de píxel sobre una imagen.  


\subsection*{Textura} 
Esta característica tiene una gran relevancia dado que es de las más usadas al momento de identificar objetos en regiones de interés en fotografías aéreas, micrográficas y de satélite y en el presente trabajo, al identificar las muestras de los árboles.

\bigskip
\section*{Descriptores de características locales}
Estas cuantifican globalmente una imagen, sin embargo, para poder determinar las características que cuantifican localmente las regiones de una imagen es necesario determinar que descriptor es el mejor para describir los puntos de interés de una imagen completa o los puntos de interés de cierta región de la imagen. \\

\begin{description}
\footnotesize \item[Escalamiento:]{Transforma los datos de las características en rangos específicos de cero a uno.}
\item[Normalización:]{Desplaza y re-escala valores para alcanzar un rango entre cero y uno.}

\item[Escala invariante (SIFT):]{Extrae la información y adecua en comparaciones.}

\item[Acelarado robusto (SURF):]{Toma un vecino al rededor del punto seleccionado en la imagen y es dividido en sub-regiones para cada sub-región.}

\item[Característica de diferencias en forma de cadena binaria (BRIEF):]{Orientación y menor numero de diferencias a su alrededor.}

\item[ORB* Rotada y orientada rápida:]{Determina estos puntos clave de una imágen.}
\end{description}

\section*{Solución propuesta}
\subsubsection*{Fase de recolección de muestras}
La primera fase en el desarrollo de la solución propuesta sería recolectar muestras de el objeto(s) a identificar por medio del aprendizaje máquina. Si bien es necesario tener una gran cantidad de muestras para que el presente trabajo tenga una perspectiva más amplia de lo que se necesita reconocer, también hay que considerar que se necesita información que contenga la menor cantidad de información no útil dado que esto podría sobre entrenar al modelo que se encargue de la clasificación.

\subsubsection*{Fase de procesamiento  de muestras}
Una vez recolectadas las muestras con información relevante, se procede a  entrenar a el modelo con esa información para que sea en fases posteriores este sea capaz de entender y clasificar donde estén presentes las especies arbóreas almacenadas en el modelo.

\subsubsection*{Fase de entrenamiento}
Originalmente se conocen las distintas especies arbóreas de la colección o conjunto de imágenes, pero al momento de clasificar, el algoritmo encargado de recorrer la carpeta que contiene las muestras útiles  necesita conocer que imágenes se van a tomar en cuenta. 

\subsubsection*{Fase de detección}
 En esta fase se utilizan las características globales de extracción de características de la sección 2.3 donde se hace uso del modelo clasificador de {\em bosque aleatorio} (inglés: random forest)\footnotemark, donde se establece un valor estimado de árboles por cada muestra donde se vaya a probar el modelo, en la investigación se va a utilizar un valor de 100.  
\footnotetext{Clasificador de múltiples decisiones que funciona en conjunto.}

\subsubsection*{Fase de combinación}
La fase de combinación trabaja indirectamente con las muestras para poder ver los resultados en una muestra con su contenido original. Para realizar una comparación, primero se necesita obtener una muestra del directorio de muestras original donde se pueda apreciar la información no útil en ella, posteriormente se necesita obtener la muestra con las especies arbóreas detectadas en ella (producto de la fase de detección).\\ 


%----------------------------------------------------------------------------------------
%	RESULTS 
%----------------------------------------------------------------------------------------
\vspace*{-25mm}
\section*{Resultados}
\begin{description}
\item[Experimento $A$: Misma cantidad de especies por tamaño de clase.]{El número de especies por tamaño de clase determinará que tantas imágenes podrían ser consideradas por el modelo a la hora de hacer el entrenamiento, por lo que dependiendo de este, el modelo puede tener una mejor o peor predicción en la fase de detección.}

\item[Experimento $B$:  Cantidad total de especies por tamaño de clase.]{Debido a que no todas las especies de árboles generan la misma cantidad de muestras, se van a utilizar las muestras totales de cada clase para determinar si esto repercute de forma positiva en el desarrollo de este experimento.}

\item[Experimento $C$: Misma cantidad de especies utilizando espejos de muestras.]{En el experimento de cantidad total de especies por tamaño de clase, se utilizaban la cantidad total de muestra generadas para cada clase, sin embargo, el impacto que pueda tener la misma cantidad de especies para cada clase considerando la especie que obtuvo más muestras generadas (\texttt{pino} con 8859 muestras) puede repercutir de cierta manera en que también se haga una mejor detección de esta, es por esta razón que se van a utilizar reflejos de las clases que tengan una cantidad de muestras menor a la de pino para completar esas especies faltantes.}

\item[Experimento $D$: Umbralización.]{En la umbralización se considera la porción de píxeles admitidos al momento de generar los rectángulos que posteriormente serán utilizados durante la fase de entrenamiento del modelo de la solución propuesta, por tanto, se comparan tres niveles distintos de umbral  (0.15\%, 0.25\%, 0.50\%) para comparar el que mejor resultados proporciona, mismo que servirá para definir el umbral ideal en otros experimentos.} 

\item[Experimento $E$: Límite de píxeles ausentes.]{El límite de píxeles ausentes determina que a menor número de píxeles ausentes, detectará menos especies arbóreas. Este limite debe ser un valor considerable debido a que si se utiliza un valor bastante alto puede detectar zonas que no corresponden a una especie arbórea correcta, en caso contrario, detectaría una menor cantidad de especies en las muestras. En este caso, se van a utilizar tres porcentajes de píxeles ausentes (0.75\%, 0.80\% y 0.85\%).}
\end{description}

\begin{center}
\begin{tabular}{|c|c|c|c|}
			\hline
			 \textbf{Experimento} & \textbf{Pino} & \textbf{Encino} & \textbf{Abies}\\
			\hline
			Experimento A & 67.3 & 18.4 & 14.4\\
			\hline
			Experimento B & 50.2 & 41.6 & 8.2\\
			\hline
			Experimento C & 58.1 & 34.3 & 7.7\\
			\hline
			Experimento D (0.75) & 66 & 17 & 17\\
			\hline
			Experimento D (0.80) & 62 & 24 & 14\\
			\hline
			Experimento D (0.85) & 18 & 15 & 18\\
			\hline
			Experimento E (0.75) & 70 & 20 & 10\\
			\hline
			Experimento E (0.80) & 63 & 25 & 12\\
			\hline
			Experimento E (0.85) & 66 & 22 & 12\\
			\hline
		\end{tabular}
\end{center}

\vspace*{-5mm}
\section*{Conclusiones}
El objetivo de desarrollar el inventario forestal por medio de la visión computacional, fue para que se puedan analizar las muestras que se utilicen en la solución propuesta por medio del aprendizaje máquina, que esta aprenda, lea, analice y por último, sea posible etiquetar y contar las especies detectadas a lo largo de una zona forestal. No obstante, dependerá en gran medida de los parámetros utilizados el alcance que tenga la ejecución de la solución, para efectos prácticos, en desarrollo de la tesis sugiere utilizar algunos valores con buen resultado tanto para el umbral adaptativo así como para los píxeles admitidos por muestra analizada.


 %----------------------------------------------------------------------------------------
%	REFERENCES
%----------------------------------------------------------------------------------------

%\nocite{*} % Print all references regardless of whether they were cited in the poster or not
%\bibliographystyle{plain} % Plain referencing style
%\bibliography{sample} % Use the example bibliography file sample.bib
%

\end{multicols}
\end{document}