%Resumen

\chapter{Resumen}
\markboth{Resumen}{}

{\setlength{\leftskip}{10mm}
\setlength{\parindent}{-10mm}

\autor.

Candidato para obtener el grado de \grado\orientacion.

\uanl.

\fime.

Título del estudio: \textsc{\titulo}.

\noindent Número de páginas: \pageref*{lastpage}.}

%%% Comienza a llenar aquí
\paragraph{Objetivos y método de estudio:}
En la tesis se realiza el análisis de inventarios forestales usando las muestras previamente generadas por el recorrido de un dron, esto con el fin de generar un inventario forestal más preciso y menos costoso, haciendo uso de herramientas que faciliten el proceso de recorrer manualmente todo un sector forestal.

\paragraph{Contribuciones y conlusiones:}
Durante el desarrollo de la tesis se explica el funcionamiento del algoritmo generado durante la investigación, el cual es capaz de desarollar de detectar las especies arbóreas en las muestras obtenidas. Este algoritmo propone una solución eficiente y precisa en comparación de las soluciones que actualmente se tienen.

\bigskip\noindent\begin{tabular}{lc}
\vspace*{-2mm}\hspace*{-2mm}Firma del asesor: & \\
\cline{2-2} & \hspace*{1em}\asesor\hspace*{1em}
\end{tabular}


